\section{Desarrollo}

La practica consta en diseñar un programa que reconozca los lexemas de:

\begin{itemize}
\item Números enteros con y sin signo.
\item Números decimales con y sin signo.
\end{itemize}

Ademas, un programa que reconozca las gramáticas para las siguientes formas:

\begin{itemize}
\item Operaciones matemáticas.
\item Potencia como función.
\end{itemize}

Unir ambos programas para que el primero alimente de \textit{tokens} al segundo.

\subsection{Analizador Léxico}

Con los ficheros de entrada (extensión \textit{.l}), Flex genera como salida un fichero fuente en C llamado \textbf{lex.yy.c}, que define una rutina \textbf{yylex()}. Este fichero se compila y se enlaza con la librería \textbf{-lfl} para producir un ejecutable. Dentro del fichero de entrada están definidas las expresiones regulares que cumplan con las especificaciones de la practica, estas expresiones las llamaremos \textbf{definiciones}:

\begin{itemize}
\item Expresión regular que detecta nombres de variables:
\begin{equation}
[a-zA-Z 0-9@\setminus.\_-]	
\end{equation}
\item Expresión regular que detecta números enteros con y sin signo:
\begin{equation}
[1-9]\{1\}[0-9]*|([0]\{1\})
\end{equation}
\item Expresión regular que detecta números decimales con y sin signo:
\begin{equation}
(0\setminus.0)\{1\}|(([1-9]\{1\}[0-9]*)(\setminus.([0-9]*[0-9][0-9]*)))|(\setminus.0)|(0?\setminus.[0-9]*[1-9][0-9]*)
\end{equation}
\item Expresión regular que detecta operaciones matemáticas:
\begin{equation}
"+"|"*"|"/"|"-"|"^"
\end{equation}
\item Expresión regular que detecta operaciones paréntesis:
\begin{equation}
\setminus(|\setminus)
\end{equation}
\item Expresión regular que detecta comas:
\begin{equation}
","
\end{equation}
\item Expresión regular que detecta potencias:
\begin{equation}
[pP][oO][wW]
\end{equation}
\end{itemize}

Una ves hayamos creado las \textbf{definiciones} crearemos las \textbf{reglas} en el mismo archivo:

\begin{lstlisting}
{INT} 		{
			printf("Entero %s\n",yytext);
			yylval.entero = atoi(yytext);
            return (ENTERO);
			}

{DECIMAL} 	{
			printf("Decimal %s\n",yytext);
			yylval.flotante = atof(yytext);
            return (DECIMAL);
        	}

{operadores}	 	{
					printf("Signo op\n");
                	return (yytext[0]);
                	}

{parentesis}	{
				printf("Parentesis\n");
				return (yytext[0]);
				}

{coma}			{
				printf("Coma\n");
				return (yytext[0]);
				}

{cadena}		{
				printf("Cadena: %s\n", yytext);
				yylval.cadena = malloc(sizeof(char)*(yyleng-1)); 
    			memcpy (yylval.cadena, yytext+1, yyleng-2);
    			yylval.cadena[yyleng] = 0;                
				return (CAD);
				}

"\n"            {
				printf("Salto de linea\n");
                return (yytext[0]);
                }
%%
\end{lstlisting}

\pagebreak

Finalmente, el código completo es el siguiente:

\begin{lstlisting}
%{
#include "syntactic.tab.h"
%}


INT [1-9]{1}[0-9]*|([0]{1})
DECIMAL (0\.0){1}|(([1-9]{1}[0-9]*)(\.([0-9]*[0-9][0-9]*)))|(\.0)|(0?\.[0-9]*[1-9][0-9]*)
CHARACTERS [a-zA-Z 0-9@\._-]	
cadena \"{CHARACTERS}*\"
operadores "+"|"*"|"/"|"-"|"^"
parentesis \(|\)
coma ","
pow [pP][oO][wW]

%%

{INT} 		{
			printf("Entero %s\n",yytext);
			yylval.entero = atoi(yytext);
            return (ENTERO);
			}

{DECIMAL} 	{
			printf("Decimal %s\n",yytext);
			yylval.flotante = atof(yytext);
            return (DECIMAL);
        	}

{operadores}	 	{
					printf("Signo op\n");
                	return (yytext[0]);
                	}

{parentesis}	{
				printf("Parentesis\n");
				return (yytext[0]);
				}

{coma}			{
				printf("Coma\n");
				return (yytext[0]);
				}

{pow}			{
				printf("Potencia\n");
				return (POW);
				}


{cadena}		{
				printf("Cadena: %s\n", yytext);
				yylval.cadena = malloc(sizeof(char)*(yyleng-1)); 
    			memcpy (yylval.cadena, yytext+1, yyleng-2);
    			yylval.cadena[yyleng] = 0;                
				return (CAD);
				}

"\n"            {
				printf("Salto de linea\n");
                return (yytext[0]);
                }
%%
\end{lstlisting}

\subsection{Analizador Sintáctico}

El código del analizador sintáctico es el siguiente:

\begin{lstlisting}
%{
#include <stdio.h>
#include <stdlib.h>
#include <stdbool.h>
extern int yylex(void);
void yyerror(char *s);
int contarDigitos();
char *potenciarCadena(char *cadena, int num, int digitos);
char *concatenarCadenas(char *texto1, char *texto2);
double potency(double base, double exponent);
char *convertToString(int number);
char *falsoOVerdadero(unsigned int numero);
%}
             
/* Declaraciones de BISON */
%union{
	int   entero;
	float flotante;
	char* cadena;
	unsigned int boolean;
}

%token <entero> ENTERO
%token <flotante> DECIMAL
%token <cadena> CAD
%token POW 
%token IF

%type <entero> exp
%type <flotante> expFloat
%type <cadena> texto
%type <boolean> condition

          
%left '+'  '-' 
%left '^'
%left '*' '/'
%left '('  %left ')'

             
/* Gramatica */
%%
             
input:   /* cadena vacia */
        | input line 
;

line:   '\n'
		| exp '\n'  { printf ("\tresultado: %d\n\n", $1); }
		| expFloat '\n' { printf("\tflotante: %f\n", $1);}
        | texto '\n' { printf ("\tresultado: %s\n\n", $1); } 
        | condition '\n'{printf("\t%s\n", falsoOVerdadero($1));}
;
            
exp: 	ENTERO { $$ = $1;}
		| exp '+' exp        { $$ = $1 + $3;    }
		| exp '*' exp        { $$ = $1 * $3;	}
		| exp '/' exp        { $$ = $1 / $3;	}
		| exp '-' exp		 { $$ = $1 - $3;	}	
		| '-' exp			 { $$ = -$2;		}
		| '(' exp ')'		 { $$ = $2;			}
		| POW '(' exp ',' exp ')'	{ $$ = potency($3, $5);}
		| POW '(' exp ',' expFloat ')'	{ 
			int exponente  = $5;
			$$ = potency($3, exponente);
		}
;

expFloat:	DECIMAL { $$ = $1;}
			| expFloat '-' exp		 		{ $$ = $1 - $3;	}
			| exp '-' expFloat		 		{ $$ = $1 - $3;	}
			| expFloat '-' expFloat			{ $$ = $1 - $3;	}
			| expFloat '+' exp		 		{ $$ = $1 + $3;	}
			| exp '+' expFloat		 		{ $$ = $1 + $3;	}
			| expFloat '+' expFloat			{ $$ = $1 + $3;	}
			| expFloat '*' exp		 		{ $$ = $1 * $3;	}
			| exp '*' expFloat		 		{ $$ = $1 * $3;	}
			| expFloat '*' expFloat			{ $$ = $1 * $3;	}
			| expFloat '/' exp		 		{ $$ = $1 / $3;	}
			| exp '/' expFloat		 		{ $$ = $1 / $3;	}
			| expFloat '/' expFloat			{ $$ = $1 / $3;	}
			| '-' expFloat					{ $$ = -$2;	   	}
			| '(' expFloat ')'		 		{ $$ = $2;	   	}
			| POW '(' expFloat ',' exp ')'	{ $$ = potency($3, $5);}
			| POW '(' expFloat ',' expFloat ')'	{ 
				int exponente  = $5;
				$$ = potency($3, exponente);
			}
;
			
texto: 	CAD {printf("cadena\n");}
		| texto '+' texto  {
			$$ = concatenarCadenas($1,$3);
		}
		| texto '+' exp {
			char *str = convertToString($3);
			$$ = concatenarCadenas($1, str);
		}
		| '(' texto ')'		{ $$ = $2; }
		| texto '^' exp {
					int nDigitos = contarDigitos($1);
					char *potencia = potenciarCadena($1, $3, nDigitos);
					$$ = potencia;
				}
;

condition: IF '(' exp '>' exp ')' ';' { 
				unsigned int num = ($3 > $5)? 1:0; 
				$$ = num;
			}
			| IF '(' exp '<' exp ')' ';' { 
				unsigned int num = ($3 < $5)? 1:0; 
				$$ = num;
			}
			| IF '(' expFloat '>' expFloat ')' ';' { 
				unsigned int num = ($3 > $5)? 1:0; 
				$$ = num;
			}
			| IF '(' expFloat '<' expFloat ')' ';' { 
				unsigned int num = ($3 < $5)? 1:0; 
				$$ = num;
			}
			| IF '(' expFloat '>' exp ')' ';' { 
				unsigned int num = ($3 > $5)? 1:0; 
				$$ = num;
			}
			| IF '(' expFloat '<' exp ')' ';' { 
				unsigned int num = ($3 < $5)? 1:0; 
				$$ = num;
			}
			| IF '(' exp '>' expFloat ')' ';' { 
				unsigned int num = ($3 > $5)? 1:0; 
				$$ = num;
			}
			| IF '(' exp '<' expFloat ')' ';' { 
				unsigned int num = ($3 < $5)? 1:0; 
				$$ = num;
			}
			| IF '('condition ')' ';' { 
				unsigned int num = ($3 == 1)? 1:0; 
				$$ = num;
			}

			

 
%%

int main() {
  yyparse();
}
             
void yyerror (char *s)
{
  printf("--%s--\n", s);
}
            
int yywrap()  
{  
  return 1;  
}  

int contarDigitos(char *cad){
	int contador = 0;
	while(*cad != 0){
		contador++;
		cad++;
	}
	return contador;
}

char *potenciarCadena(char *cadena, int num, int digitos){
	int i = 0;
	int j = 0;
	int asignacion = 0;
	int size = (digitos*num) + 1;
	char *str = malloc(sizeof(char)*(size));
	for(i; i < num; i++){
		for(j=0; j < digitos; j++){
			*(str+asignacion) = *(cadena+j);
			asignacion++;
		}
	}
	str[size-1] = 0;
	return str;
}

double potency(double base, double exponent){
	double result = 1;	
	while (exponent != 0) {
        result *= base;
        --exponent;
    }
    return result;
}

char *convertToString(int number){
	char buffer[50];
	int i = 0;

	bool isNeg = number < 0;

	unsigned int n1 = isNeg ? -number : number;

	while(n1!=0)
	{
	    buffer[i++] = n1%10+'0';
	    n1=n1/10;
	}

	if(isNeg)
	    buffer[i++] = '-';

	buffer[i] = '\0';

	for(int t = 0; t < i/2; t++)
	{
	    buffer[t] ^= buffer[i-t-1];
	    buffer[i-t-1] ^= buffer[t];
	    buffer[t] ^= buffer[i-t-1];
	}

	if(number == 0)
	{
	    buffer[0] = '0';
	    buffer[1] = '\0';
	}  

	int size = contarDigitos(buffer) +1;
	char *s = malloc(sizeof(char)*(size));
	for (i=0; buffer[i] != 0; i++){
		*(s+i) = buffer[i];
	}
	*(s+i) = '\0';
	return s;

}

char *concatenarCadenas(char *cad1, char *cad2) {
	int texto1 = contarDigitos(cad1);
	int texto2 = contarDigitos(cad2);
	char *s = malloc(sizeof(char)*(texto1 + texto2 + 1));
	int i=0;
	for(i;i < texto1; i++){
		*(s+i) = *cad1++;
	}
	for(i;i < (texto1 + texto2); i++){
		*(s+i) = *cad2++;
	}
	return s;
}

char *falsoOVerdadero(unsigned int numero){
	char *boolean = (numero == 0)? "false" : "true";
	return boolean;
}
\end{lstlisting}

\pagebreak