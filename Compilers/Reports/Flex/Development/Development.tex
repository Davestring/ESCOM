\section{Desarrollo}

La practica consta en diseñar las expresiones regulares en \textbf{Flex} que reconozcan los siguientes patrones:

\begin{itemize}
\item Nombres de Variables.
\item Números enteros con y sin signo.
\item Números decimales con y sin signo.
\item Potencias de números reales.
\item Operaciones matemáticas como suma, resta, división, multiplicación y modulo.
\end{itemize}

\subsection{Archivo Lex Source Format}

Con los ficheros de entrada (extensión \textit{.l}), Flex genera como salida un fichero fuente en C llamado \textbf{lex.yy.c}, que define una rutina \textbf{yylex()}. Este fichero se compila y se enlaza con la librería \textbf{-lfl} para producir un ejecutable. Dentro del fichero de entrada están definidas las expresiones regulares que cumplan con las especificaciones de la practica, estas expresiones las llamaremos \textbf{definiciones}:

\begin{itemize}
\item Expresión regular que detecta nombres de variables:
\begin{equation}
[a-zA-Z\_]\{1\}[a-zA-Z0-9\_]*
\end{equation}
\item Expresión regular que detecta números enteros con y sin signo:
\begin{equation}
\setminus-?[0]\{1\}|(\setminus-?)[1-9]\{1\}[0-9]*
\end{equation}
\item Expresión regular que detecta números decimales con y sin signo:
\begin{equation}
(\setminus-?)((0?\setminus.[0-9]*[1-9][0-9]*)|(([1-9]\{1\}[0-9]*)((\setminus.([0-9]*[1-9][0-9]*))|(\setminus.0)))|(0\setminus.0)\{1\})
\end{equation}
\item Expresión regular que detecta potencias:
\begin{equation}
[p|P][o|O][w|W]\setminus((\{integer\}|\{decimal\}),(\{integer\}|\{decimal\})\setminus)
\end{equation}
\item Expresión regular que detecta la operación modulo:
\begin{equation}
mod\setminus((\{integer\}|\{decimal\}),(\{integer\}|\{decimal\})\setminus)
\end{equation}
\end{itemize}

Una ves hayamos creado las \textbf{definiciones} crearemos las \textbf{reglas} en el mismo archivo:

\begin{itemize}
\item Operaciones matemáticas:
\begin{lstlisting}
\= {printf("equal ");}
\+ {printf("sum ");}
\- {printf("minus ");}
\* {printf("multiply ");}
\/ {printf("division ");}
\end{lstlisting}
\item Variables, números, potencia y modulo:
\begin{lstlisting}
{variable} {printf("variable ");}
{integer} {printf("integer_number ");}
{decimal} {printf("decimal_number ");}
{pow} {printf("potency ");}
{mod} {printf("module ");}
\end{lstlisting}
\end{itemize}

\pagebreak

Finalmente, el código completo es el siguiente:

\begin{lstlisting}
variable [a-zA-z_]{1}[a-zA-Z0-9_]*
integer \-?[0]{1}|(\-?)[1-9]{1}[0-9]*
decimal (\-?)((0?\.[0-9]*[1-9][0-9]*)|(([1-9]{1}[0-9]*)((\.([0-9]*[1-9][0-9]*))|(\.0)))|(0\.0){1})
pow [p|P][o|O][w|W]\(({integer}|{decimal}),({integer}|{decimal})\)
pow1 ({decimal}|{integer})\^({decimal}|{integer})
mod mod\(({integer}|{decimal}),({integer}|{decimal})\)
%%
{variable} {printf("variable ");}
{integer} {printf("integer_number ");}
{decimal} {printf("decimal_number ");}
{pow} {printf("potency ");}
{pow1} {printf("potency ");}
{mod} {printf("module ");}
\= {printf("equal ");}
\+ {printf("sum ");}
\- {printf("minus ");}
\* {printf("multiply ");}
\/ {printf("divition ");}
\end{lstlisting}

\pagebreak